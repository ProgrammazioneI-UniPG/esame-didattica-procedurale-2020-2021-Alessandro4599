\documentclass[addpoints,11pt]{exam}
\usepackage[top=0.5in, bottom=0.5in, left=0.5in, right=0.5in]{geometry}
\usepackage[utf8]{inputenc}
\usepackage{listings}
\usepackage{color,graphicx}
\usepackage{multicol}
\usepackage{MnSymbol}



\definecolor{codegreen}{rgb}{0,0,0}
\definecolor{codegray}{rgb}{0,0,0}
\definecolor{codepurple}{rgb}{0,0,0}
\definecolor{backcolour}{rgb}{1,1,1}

\lstdefinestyle{mystyle}{
	backgroundcolor=\color{backcolour},   
	commentstyle=\color{codegreen},
	keywordstyle=\color{black},
	numberstyle=\tiny\color{codegray},
	stringstyle=\color{codepurple},
	basicstyle=\footnotesize,
	breakatwhitespace=false,         
	breaklines=true,                 
	captionpos=b,                    
	keepspaces=true,                 
	numbers=left,                    
	numbersep=5pt,                  
	showspaces=false,                
	showstringspaces=false,
	showtabs=false,                  
	tabsize=2
}

\lstset{style=mystyle}

\pagestyle{empty}
 
\begin{document}
 \boxedpoints
 \pointname{~punti}
\begin{center}
\fbox{\fbox{\parbox{5.5in}{\centering
Prova scritta Didattica - Programmazione Procedurale - 17 Novembre 2020 }}}
\end{center}
 
\vspace{5mm}
 
\noindent\makebox[\textwidth]{Nome e Cognome: \rule{8cm}{.1pt} \hspace{1cm} Matricola:  \rule{5cm}{.1pt}}

%\vspace{5mm}
%\makebox[\textwidth]{Matricola:\enspace\hrulefill}
 

\begin{questions} 

\question[1] Scrivere cosa stampa il seguente programma.

\begin{minipage}[t]{0.4\linewidth}
	\begin{lstlisting}[language=C]
int main()
{
    int a,b,c,somma;
    a = 2;
    b = a++;
    c = ++a;
    somma = a+b+c;
    return 0;
}
\end{lstlisting}
\end{minipage}
\begin{minipage}[t]{0.6\linewidth}
	\makeemptybox{100pt}
\end{minipage}



\question[2]
Trova l'errore e correggi il programma.

\begin{minipage}[t]{0.60\linewidth}
	
		\begin{lstlisting}[language=C]
int main(){
    int **mat,righe,colonne,i;
    righe = colonne = 3;
    mat=(int**)malloc(sizeof(int *)*righe); 
    for(i=0;i<colonne;i++){
        mat[i]=(int*)malloc(sizeof(int)*colonne); 
    }
    //... programma...
    free(mat);  
    return 0;
}
\end{lstlisting}
\end{minipage}
\begin{minipage}[t]{0.4\linewidth}
	\makeemptybox{160pt}
\end{minipage}






\question[2] Scrivere cosa stampa il seguente programma.

\begin{minipage}[t]{0.5\linewidth}
	\begin{lstlisting}[language=C]
int main(){
    for (int i=0; i<5; i++) {
        printf(" %d \n", i);}    
    printf("\n\n");    
    for (int j=1; j<=5; ++j){
        printf(" %d \n", j);}    
    return 0;}
\end{lstlisting}
\end{minipage}
\begin{minipage}[t]{0.5\linewidth}
	\makeemptybox{120pt}
\end{minipage}


\question[2] Qual è il valore delle variabili "c" e "d"?

\begin{minipage}[t]{0.4\linewidth}
	\begin{lstlisting}[language=C]
int main(){
    int a[5] = {1,2,3,4,5};
    int c, d, *b;
    b = &a[2];
    c = *b +1;  
    d = *(b+1);  
    return 0;}
\end{lstlisting}
\end{minipage}
\begin{minipage}[t]{0.6\linewidth}
	\makeemptybox{140pt}
\end{minipage}



\question[3] Scrivere cosa stampa il seguente programma.

\begin{minipage}[t]{0.55\linewidth}
	\begin{lstlisting}[language=C]
int main(){
    int x = 5;
    int i, j;
    for (i = 1; i <= x; i++) {
        for (j = 1; j <= x - i; j++)
            printf(" ");
        for (j = 1; j <= i*2 - 1; j++)
            printf("*");
    printf("\n");
  }
  return 0;
}
\end{lstlisting}
\end{minipage}
\begin{minipage}[t]{0.45\linewidth}
	\makeemptybox{170pt}
\end{minipage}



\question[2] Trova l'errore e correggi il programma.

\begin{minipage}[t]{0.6\linewidth}
	\begin{lstlisting}[language=C]
int main(){
    int **mat,righe,colonne,i;
    righe = colonne = 3;
    mat=(int**)malloc(sizeof(int)*righe); 
    for(i=0;i<colonne;i++){
        mat[i]=(int*)malloc(sizeof(int)*colonne); 
         }
    return 0;
}
\end{lstlisting}
\end{minipage}
\begin{minipage}[t]{0.4\linewidth}
	\makeemptybox{150pt}
\end{minipage}








\question[1] Scrivere cosa stampa il seguente programma.

\begin{minipage}[t]{0.4\linewidth}
	\begin{lstlisting}[language=C]
int main (){
  int *b, *c, a;
  a = 2;
  b = &a;
  a = 0;
  *b = *b + 1;
  c = &a;
  printf("%d \n", a);      
  printf("%d \n", *b);     
  printf("%d \n", *c);     
  return 0;
}
\end{lstlisting}
\end{minipage}
\begin{minipage}[t]{0.6\linewidth}
	\makeemptybox{150pt}
\end{minipage}



\question[2]
Cosa stampa il seguente programma.

\begin{minipage}[t]{0.4\linewidth}
	
		\begin{lstlisting}[language=C]
int main(){
    int x = 5;
    int y = 3;
    fun(x,y);
    printf("x : %d \n", x);  
    printf("y : %d \n", y);  
    return 0;
}

void fun(int x, int y){
    x++;
    ++y;
    y = y + x++;
}
\end{lstlisting}
\end{minipage}
\begin{minipage}[t]{0.6\linewidth}
	\makeemptybox{160pt}
\end{minipage}






\question[2] Qual è il valore della variabile "somma".

\begin{minipage}[t]{0.6\linewidth}
	\begin{lstlisting}[language=C]
int main()
{
    enum mesi { gen, feb, mar, apr, mag = 7, giu, lug, ago, set, ott, nov, dic };
    int somma = feb + lug + ott;
    printf("somma = %d \n", somma); 
    return 0;
}
\end{lstlisting}
\end{minipage}
\begin{minipage}[t]{0.4\linewidth}
	\makeemptybox{170pt}
\end{minipage}


\question[2] Trova l'errore e correggi.

\begin{minipage}[t]{0.6\linewidth}
	\begin{lstlisting}[language=C]
#include <stdio.h>
#include <stdlib.h>
struct lista {
    int val;
    struct lista * successivo;
};
int main(){
  struct lista * a = malloc(sizeof(struct lista));
  a->val = 5;
  a->successivo->val = 3;
  int somma = a->val + a->successivo->val;
  printf("%d \n ", somma);
}
\end{lstlisting}
\end{minipage}
\begin{minipage}[t]{0.4\linewidth}
	\makeemptybox{170pt}
\end{minipage}



\question[3] Scrivere cosa stampa il seguente programma.

\begin{minipage}[t]{0.55\linewidth}
	\begin{lstlisting}[language=C]
int main()
{
    int a[5]= {1,2,3,4};
    for(int i=0; i<=a[i]; i++){
        printf("numero %d\n", i);
        if(i != 0)
        i= i+a[i-1];
        if(i==3)
        continue;
    }
    return 0;
}
\end{lstlisting}
\end{minipage}
\begin{minipage}[t]{0.45\linewidth}
	\makeemptybox{170pt}
\end{minipage}



\question[3] Scrivere cosa stampa il seguente programma.

\begin{minipage}[t]{0.5\linewidth}
	\begin{lstlisting}[language=C]
int main()
{
   int a= 5;
   short int b=2;
   float c = 3.2;
   double d= 4.7;
   int e = (char) a+b+c+d;   
   printf("%d\n", e);
    return 0;
}
\end{lstlisting}
\end{minipage}
\begin{minipage}[t]{0.5\linewidth}
	\makeemptybox{140pt}
\end{minipage}




\question[2] Cosa stampa il seguente programma.

\begin{minipage}[t]{0.55\linewidth}
	\begin{lstlisting}[language=C]
int main()
{
    int a[5]= {6,1,1,6,1};
    for(int i=0; i<=a[i]; i++){
        if(i>a[i])
         continue;
        printf("numero %d\n", i);
    }
    return 0;
}
\end{lstlisting}
\end{minipage}
\begin{minipage}[t]{0.45\linewidth}
	\makeemptybox{150pt}
\end{minipage}



\question[2] Quanto vale la variabile "a" alla fine del programma?

\begin{minipage}[t]{0.6\linewidth}
	\begin{lstlisting}[language=C]
int main()
{
   int a= (0 && 1 || 2 && sizeof(int)? 0 : 5);
   printf("%d", a);
    return 0;
}
\end{lstlisting}
\end{minipage}
\begin{minipage}[t]{0.4\linewidth}
	\makeemptybox{120pt}
\end{minipage}



\question[2] Cosa stampa il seguente programma.

\begin{minipage}[t]{0.5\linewidth}
	\begin{lstlisting}[language=C]
//ULLONG_MAX = 8446744073709551615
int main()
{
    short x = ULLONG_MAX;
    char y= 7.50;
    int c = x+y;
    printf("%d", c);
    return 0;
}
\end{lstlisting}
\end{minipage}
\begin{minipage}[t]{0.5\linewidth}
	\makeemptybox{140pt}
\end{minipage}























\end{questions}

\end{document}